\documentclass[crop=false]{standalone}
\begin{document}
	\section{Expected Result}
    我們預期使用Dynamic Window Approach (DWA)方法將能夠有效地提高無人機在靜態環境中的避障能力。透過結合感知和控制,加上DWA方法能夠快速生成並選擇最優的移動速度和方向,使無人機能夠即時地避開障礙物,確保安全和順暢的飛行。
    
    具體來說,我們期望DWA方法能夠使無人機在給定隨機障礙物與靜態環境中表現良好。我們預期無人機將能夠避開障礙物並找到最優的行進路徑,同時保持穩定和平滑的飛行。此外,我們還期望DWA方法能夠在保證飛行安全性的同時,具有較高的計算效率,能夠在實時性要求下快速運行。
    
    總結來說,我們預期透過使用DWA方法,無人機將能夠避開障礙物,並以穩定的速度行向自目的地,從而提高無人機在實際應用中的運行效率和安全性,推動無人機技術在各個應用領域的廣泛應用。
\end{document}