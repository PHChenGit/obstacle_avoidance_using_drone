\documentclass[12pt, a4paper, oneside]{article}

\providecommand{\pgfsyspdfmark}[3]{}

% border setting
\usepackage[ top=2.5cm,bottom=2.5cm,left=2.5cm,right=2.5cm ]{ geometry }

% The default for LaTeX is to have no indent after sectional headings, like \chapter and \section. ()
\usepackage{indentfirst}

% http://tex.stackexchange.com/questions/28333/continuous-v-per-chapter-section-numbering-of-figures-tables-and-other-docume
% 讓圖片,表格編號自動連續編號
\usepackage{chngcntr}
\counterwithout{figure}{section}
\counterwithout{table}{section}

% This prevents placing floats before a section
\usepackage[section]{placeins}
\let\Oldsubsection\subsection
\renewcommand{\subsection}{\FloatBarrier\Oldsubsection}

% source code hightlighting
\usepackage{listings}
\lstset{
	numbers=left,
	stepnumber=1,
	firstnumber=1,
	captionpos=b,
	tabsize=2,
	basicstyle=\small,
	numberfirstline=true
}

\usepackage{graphicx}
\graphicspath{ {./figures/} }

% Insert watermark for each pages
\usepackage{background}
\backgroundsetup{
	contents={\includegraphics[]{ntut_watermark.jpg}},
	scale=0.2,
	opacity=0.4,
	angle=0
}

%\usepackage{wallpaper}
%\CenterWallPaper{.18}{ \includegraphics[]{ntut_watermark.jpg}}

% setting the page number to footer
\usepackage{fancyhdr}
\fancyhf{}
\cfoot{\thepage}
\pagestyle{fancy}
% no header and footer bar
\renewcommand{\headrulewidth}{0pt}
\renewcommand{\footrulewidth}{0pt}

% line height setting
\linespread{1.5}
\usepackage{setspace}

\usepackage{xeCJK} %打中文必備
\setCJKmainfont{Noto Sans CJK TC} %設定中文字型,而英文不去更動

\usepackage{pdfpages}

\newcommand{\englishTitle}{Obstacle Avoidance Using Dynamic Window Approach For Drone}

\newcommand{\studentEnName}{Pao-Hsun, Chen, and Yong-Jhen, Jheng}
\newcommand{\advisorEnName}{Huei-Yung Lin}


\begin{document}
	
	%\maketitle
	\begin{titlepage}
		\begin{center}
			\LARGE
			\begin{singlespace}
				\textbf{\englishTitle{}} \\[0.5cm]
			\end{singlespace}
			
			\begin{singlespace}
				\begin{tabular}{r l}
					Student     & : \studentEnName{}  \\
					Advisor  & : Dr. \advisorEnName{} \\[0.5cm]
				\end{tabular}
			\end{singlespace}
			
			\begin{singlespace}
				Department of Computer Science
				and Information Engineering
				National Taipei University of Technology\\[0.5cm]
			\end{singlespace}
			
		\end{center}
	\end{titlepage}
	
	\section{Background}
     避開障礙物是自主移動式機器人在進行自主導航時一項基礎且必不可少的技術,無論是在工廠環境、室內環境、或是戶外環境下都是不可或缺的技能,使得這些機器人可以參與我們的日常生活以及工作。而且這樣的需求不斷在增長,例如:在餐廳中負責帶領客人前往桌位的機器人,在餐廳中負責送參的機器人,在居家環境裡負責清潔工作的機器人,在戶外環境下的自動駕駛汽車。這些機器人面臨一項共同的挑戰,需要避開動態和靜態的障礙物。動態障礙物包括行人,行進中的汽機車等等。靜態障礙物包括桌椅,貨物架,貨物等等。
     
     過去的研究已經提出了各種無人機避障方法,包括基於感測器的方法、視覺導航方法和軌跡規劃方法等。然而,這些方法在處理複雜的環境和動態的障礙物時仍然存在一定的挑戰,例如計算成本高、運算效率低、對環境變化敏感等。

     本文中採用的自主移動式機器人為四軸無人機,並且假設在室內環境下避開靜態障礙物。為了解決這個問題,我們使用基於Dynamic Window Approach (DWA)解決方案動態調整機器人行進路線,使得機器人可以避開障礙物。
     
     DWA主要是用於局部路徑規劃,在本文中因為使用無人機,所以假設在固定高度下進行實驗,這使得問題可以簡化成在 2D 空間空的路徑規劃問題。DWA在速度空間中採集多組速度,並模擬機器人在這些速度下一定時間內的軌跡。在得到多組軌跡後,對這些軌跡比較,並且從中採取最佳軌跡所對應的速度來驅動機器人運動。
	\section{Methodology}
	本研究採用Dynamic Window Approach (DWA)方法來解決無人機的障礙物避障問題。DWA是一種基於運動窗口的避障方法,透過結合感知和控制,使無人機能夠在複雜環境中快速生成並選擇最優的移動速度和方向,以避開障礙物。
	
	首先,我們先隨機產生障礙物,包括障礙物的位置、大小和等訊息。然後,我們利用DWA演算法生成運動窗口,即在給定時間內,無人機可以移動的所有可能速度和方向的集合。接著,我們使用一系列評估標準(如與障礙物的最小距離、速度平滑度等)對每個窗口進行評估和排序,從中選擇最優的運動窗口作為無人機的下一步行動。
	
	最後,我們將所選擇的運動窗口轉換為適當的控制指令,以實現無人機的移動。該指令將傳遞給無人機的控制系統,例如PID控制器,來調整無人機的速度和方向,以達到避障的目的。
	
	總的來說,DWA方法能夠使無人機在複雜和動態的環境中快速且有效地避開障礙物,並且在實時性和計算效率上具有優勢。因此,我們選擇採用DWA作為本研究的主要方法,以解決無人機的障礙物避障問題。
	\section{Expect Results}
	
	
	\begin{thebibliography}{1}
		
		\bibitem{fox}
		D. Fox, W. Burgard and S. Thrun, The dynamic window approach to collision avoidance," in IEEE Robotics \& Automation Magazine, vol. 4, no. 1, pp. 23-33, March 1997, doi: 10.1109/100.580977. keywords: {Collision avoidance;Mobile robots;Robot sensing systems;Orbital robotics;Robotics and automation;Motion control;Humans;Robot control;Motion planning;Acceleration},
		
		\bibitem{missura}
		M. Missura and M. Bennewitz, "Predictive Collision Avoidance for the Dynamic Window Approach," 2019 International Conference on Robotics and Automation (ICRA), Montreal, QC, Canada, 2019, pp. 8620-8626, doi: 10.1109/ICRA.2019.8794386.
		keywords: {Trajectory;Vehicle dynamics;Dynamics;Acceleration;Collision avoidance;Robot sensing systems},
		
		\bibitem{carrio}
		A. Carrio, J. Tordesillas, S. Vemprala, S. Saripalli, P. Campoy and J. P. How, "Onboard Detection and Localization of Drones Using Depth Maps," in IEEE Access, vol. 8, pp. 30480-30490, 2020, doi: 10.1109/ACCESS.2020.2971938.
		keywords: {Drones;Three-dimensional displays;Feature extraction;Object detection;Sensors;Training;Cameras;Drone;detection;collision avoidance;depth map},
		
		\bibitem{devos}
		A. Devos, E. Ebeid and P. Manoonpong, "Development of Autonomous Drones for Adaptive Obstacle Avoidance in Real World Environments," 2018 21st Euromicro Conference on Digital System Design (DSD), Prague, Czech Republic, 2018, pp. 707-710, doi: 10.1109/DSD.2018.00009. keywords: {Drones;Collision avoidance;System recovery;Laser radar;Navigation;Signal processing algorithms;Propellers;Autonomous drone system;Adaptive obstacle avoidance;Simulation;Implementation},
		
		\bibitem{tordesillas}
		J. Tordesillas, B. T. Lopez, M. Everett and J. P. How, "FASTER: Fast and Safe Trajectory Planner for Navigation in Unknown Environments," in IEEE Transactions on Robotics, vol. 38, no. 2, pp. 922-938, April 2022, doi: 10.1109/TRO.2021.3100142.
		keywords: {Trajectory;Safety;Resource management;Optimization;Planning;Aerospace electronics;Hardware;Convex decomposition;path planning;trajectory optimization;UAV},
		
		
	\end{thebibliography}
	
\end{document}
